\documentclass{article}

% if you need to pass options to natbib, use, e.g.:
% \PassOptionsToPackage{numbers, compress}{natbib}
% before loading nips_2018

% ready for submission
\usepackage[final]{nips_2018}

% to compile a preprint version, e.g., for submission to arXiv, add
% add the [preprint] option:
% \usepackage[preprint]{nips_2018}

% to compile a camera-ready version, add the [final] option, e.g.:
% \usepackage[final]{nips_2018}

% to avoid loading the natbib package, add option nonatbib:
% \usepackage[nonatbib]{nips_2018}

\usepackage[utf8]{inputenc} % allow utf-8 input
\usepackage[T1]{fontenc}    % use 8-bit T1 fonts
\usepackage{hyperref}       % hyperlinks
\usepackage{url}            % simple URL typesetting
\usepackage{booktabs}       % professional-quality tables
\usepackage{amsfonts}       % blackboard math symbols
\usepackage{nicefrac}       % compact symbols for 1/2, etc.
\usepackage{microtype}      % microtypography

\title{Project Milestone: arXiv vs. snarXiv}

% The \author macro works with any number of authors. There are two
% commands used to separate the names and addresses of multiple
% authors: \And and \AND.
%
% Using \And between authors leaves it to LaTeX to determine where to
% break the lines. Using \AND forces a line break at that point. So,
% if LaTeX puts 3 of 4 authors names on the first line, and the last
% on the second line, try using \AND instead of \And before the third
% author name.

\author{Tyler Blanton \And Sam Kowash}
  %% examples of more authors
  %% \And
  %% Coauthor \\
  %% Affiliation \\
  %% Address \\
  %% \texttt{email} \\
  %% \AND
  %% Coauthor \\
  %% Affiliation \\
  %% Address \\
  %% \texttt{email} \\
  %% \And
  %% Coauthor \\
  %% Affiliation \\
  %% Address \\
  %% \texttt{email} \\
  %% \And
  %% Coauthor \\
  %% Affiliation \\
  %% Address \\
  %% \texttt{email} \\


\begin{document}
% \nipsfinalcopy is no longer used

\maketitle

\begin{abstract}
  We give a brief update on our project to develop a classifier that can accurately distinguish between real arXiv \texttt{hep-th} abstracts and fake abstracts generated from a context-free grammar by the program snarXiv.
  Our progress consists of three major parts: obtaining large amounts of data in an efficient way, parsing the abstract text to prepare it for analysis, and implementing a simple classification algorithm -- a naive Bayes classifier using a bag-of-words model.
  We chose this extremely simple classifier to serve as a proof of concept, though it already outperforms humans (successful classification rate is 75-80\% vs. 59\% for humans).
\end{abstract}




\section{Progress}
\label{sec:progress}
\subsection{Obtaining abstracts}
The arXiv exposes its search API through a well-documented HTTP interface, but this has limitations in the context of our project as it is configured to find no more than 30000 results, and return no more than 2000 of them at a time.
We estimate that \texttt{hep-th} has $\sim$85000 articles, and we'd ideally train on all of them, so we need a more robust data acquisition tool.
Fortunately, arXiv also furnishes an Open Archives Initiative Protocol for Metadata Harvesting (OAI-PMH) interface, which is better-suited to full-repository harvesting. We use the \texttt{pyoai}\footnote{\url{https://github.com/infrae/pyoai}} module to retrieve the full metadata set from \texttt{hep-th}, extract the abstract text from the XML structure, and feed it to the parser.

The snarXiv abstracts are even easier to obtain, since we can generate them at will.
We use a modified grammar file to produce only abstracts, then call the compiled snarxiv generator repeatedly to produce a corpus, which can then be fed to the parser.
(The snarXiv generator could of course be modified to produce the desired format in the first place, but we prefer to use as much of the original tool as possible.








\subsection{Parsing abstract text}














\subsection{Classifying abstracts}






















\section{Plans}
\subsection{Feature development}
Currently, 














\subsection{Classification}

\end{document}
