\documentclass{article}
\usepackage[margin=1.0in]{geometry}
\usepackage[utf8]{inputenc}
\usepackage{braket,amsmath,amsfonts,mathtools,graphicx,subcaption,color}

\title{Project Milestone: arXiv vs. snarXiv}
\author{Tyler Blanton, Samuel Kowash}
\date{November 15, 2018}


\begin{document}
\maketitle

\begin{abstract}
We give a brief update on our project to create a program that can accurately distinguish between real arXiv \texttt{hep-th} abstracts and fake abstracts generated by the program snarXiv.
Our progress consists of three major parts: obtaining large amounts of data in an efficient way, parsing the abstract text to prepare it for analysis, and implementing a simple classification algorithm -- a naive Bayes classifier using a bag-of-words model.
We chose this extremely simple classifier to serve as a proof of concept, though it already outperforms humans (successful classification rate is 75-80\% vs. 59\% for humans).
\end{abstract}

\section{Obtaining abstracts}



\section{Parsing abstract text}
The raw response from an arXiv api call is a bunch of text consisting of full \texttt{hep-th} papers (which often contain TeX symbols like \$ and  \textbackslash) along with several tags specifying sections as title, author, abstract, etc.
For simplicity, we are currently only focusing on the abstracts, which we pull out and organize into a list using Python's \texttt{feedparser} package.
We also store snarXiv abstracts in this form, so the rest of the parsing process is the same for the arXiv and snarXiv data.

The next step is to split the abstract text into regular words that are easily identifiable between different abstracts.
We split on all spaces and newlines, and for uniformity we make all words lowercase and strip out all punctuation.
This method works for the most part, but it leaves traces of some TeX commands; for example, it sends \texttt{\textbackslash mathbb\{Z\} $\to$ mathbbz}.
In the future we may try a more sophisticated approach where we deal with TeX commands at the beginning before stripping all punctuation, but for now this issue is only a minor nuisance.
In any case, the output of our parsing procedure is two lists of lists (one arXiv, one snarXiv), where each inner list is comprised of formatted words from a single abstract.



\section{Classifying abstracts}




\end{document}