\documentclass{article}
\usepackage{amsthm}
\usepackage{amssymb}
\usepackage{diagrams}
\pagestyle{empty}
\setlength\textwidth{3.9in}

\newtheorem*{theorem}{Theorem}
\newcommand\Z{\mathbb{Z}}
\newcommand\Q{\mathbb{Q}}
\newcommand\R{\mathbb{R}}
\newcommand\C{\mathbb{C}}
\newcommand\lt{<}
\newcommand\gt{>}
\newcommand\squarecode[8]{
      #1        & \rTo^{#5} & #2       \\
      \dTo^{#6} &           & \dTo^{#7}\\
      #3        & \rTo^{#8} & #4\\
}
\newcommand\bottomadd[5]{
\dTo^{#1} & & \dTo^{#2}\\
#3 & \rTo^{#4} & #5\\
}
\newcommand\onebyone[8]{
\begin{diagram}
\squarecode{#1}{#2}{#3}{#4}{#5}{#6}{#7}{#8}
\end{diagram}
}
\newcommand\nothing{}
\newcommand\oneldots{
\begin{diagram}
\cdots\\
\\
\cdots\\
\end{diagram}
}
\newcommand\twoldots{
\begin{diagram}
\cdots\\
\\
\cdots\\
\\
\cdots\\
\end{diagram}
}
\newcommand\threeldots{
\begin{diagram}
\cdots\\
\\
\cdots\\
\\
\cdots\\
\\
\cdots\\
\end{diagram}
}
\newcommand\rightonearrow[2]{
&\rTo^{#1} & #2\\
}
\newcommand\bottomrightadd[3]{
& & \dTo^{#1}\\
& \rTo^{#2} & #3\\
}
\newcommand\rightaddone[5]{
\begin{diagram}
\rightonearrow{#1}{#2}
\bottomrightadd{#3}{#4}{#5}
\end{diagram}
}
\newcommand\rightaddtwo[8]{
\begin{diagram}
\rightonearrow{#1}{#2}
\bottomrightadd{#3}{#4}{#5}
\bottomrightadd{#6}{#7}{#8}
\end{diagram}
}
\newcommand\rightaddonedots[2]{
\begin{diagram}
&\rTo^{#1}&\cdots\\
& & &\\
& \rTo^{#2} & \cdots\\
\end{diagram}
}
\newcommand\rightaddtwodots[3]{
\begin{diagram}
&\rTo^{#1}&\cdots\\
& & &\\
& \rTo^{#2} & \cdots\\
& & &\\
& \rTo^{#3} & \cdots\\
\end{diagram}
}
