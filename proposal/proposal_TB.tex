\documentclass{article}
\usepackage[utf8]{inputenc}
\usepackage{amsmath,amsfonts,color}

\title{arXiv or snarXiv?}
\author{Samuel Kowash, Tyler Blanton}
\date{October 25, 2018}

\begin{document}

\maketitle
The arXiv is a free online repository of preprints of scientific papers (many of which are also published in scientific journals) from fields such as mathematics, physics, and computer science.
Submissions are moderated to ensure only legitimate papers are posted, making it an extremely valuable resource for scientists in many disciplines.
% contains over a million papers (many of which are also published in scientific journals).

The snarXiv is a computer program created by David Simmons-Duffin that randomly generates titles and abstracts for high-energy theoretical physics (HEP-th) papers.
He used it make the game ``arXiv vs. snarXiv," in which readers are presented with two title/abstract pairs -- one from the HEP-th sector of arXiv, the other generated with snarXiv -- and are asked to choose which one is real.
After 750,000 guesses, the success rate was only 59\%; i.e., people mistakenly chose the randomly generated snarXiv title/abstract to be the real one 41\% of the time, which is pretty remarkable (and amusing).

Our goal is to use machine learning techniques to write a program that accurately classifies HEP-th title/abstract pairs as being real (arXiv) or fake (snarXiv).
We would like to explore how different learning algorithms perform; for example, we predict that sequence models like convolutional neural networks will outperform n-gram models like logistic regression due to the grammatical syntax that snarXiv uses, and we would like to test this.
Although the scope our project is restricted to the admittedly impractical field of faux physics literature, the problem of distinguishing real vs. simulated writing does have real-world applications such as spam-filtering. 


The arXiv has an API%
\footnote{https://arxiv.org/help/api/index}
for downloading text samples, and Simmons-Duffin has made his snarXiv code available on github.%
\footnote{https://github.com/davidsd/snarxiv}
%





Depending on how things go, we may try to add viXra to the mix.
viXra is like arXiv except that all submissions are accepted, which is why a significant percentage of its papers are low-quality and/or factually inaccurate.
As far as we know there is no API for viXra though, so getting data from it may not be worth the effort.


Supervised learning: logistic regression, support vector machine, 

\end{document}